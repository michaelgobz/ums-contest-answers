\documentclass[a4paper, 12pt]{article}
\usepackage[T1]{fontenc}
\usepackage[utf8]{inputenc}
\usepackage{geometry}
\usepackage{fancyhdr}
\usepackage{amssymb}
\usepackage{ragged2e}
\usepackage{mathtools}

\pagestyle{fancy}
\fancyhf{}
\rhead{Makerere Univeristy | Kampala }
\lhead{\bfseries Michael Goboola  09/29/2000  Male }
\rfoot{}


\begin{document}

\begin{center}
    \fontsize{24pt}{10pt}\selectfont
    \textsc{\textbf{Attempted}}
\end{center}

\section{Question 1}

$a  \in  \mathbb{R}, b  \in  \mathbb{R}$ then by  the mapping $g$ , $g(a)  \in \mathbb{R}$ and
$g(b) \in \mathbb{R}$. Suppose $a,b$ $\in \mathbb{N}$  and $\mathbb{N} \subset \mathbb{R}$, 
then by the principle of recursive definition there exists a function $\phi$ such that
$\phi(a + b) = \phi(a) + \phi(b)$ and $\phi(ab) = \phi(a)\phi(b)$.
If $a = b$ , then $g(a)^2 \equiv g(a)g(a) \equiv g(a^2)$  and $g(a^2 + a + g(a)) \equiv g(a^2)  + g(a) + g(g(a))$.
Then proceeding with contradiction taking from the problem we have

   \begin{justify} 
    \begin{center}
    $g(a^2 + b + g(b)) = 2b + g(a)^2$ \newline
    taking $b = a, \forall a,b \in \mathbb{N} \subset\mathbb{R}$ \newline 
    by the principle of recursive definition \newline
    $g(a)g(a) + g(a) + g(g(a)) = 2a + g(a)g(a)$ \newline
    $g(a) + g(g(a)) = 2a$ \newline
    \end{center}
   \end{justify}

but since $a \in \mathbb{N}$ and $\mathbb{N} \subset \mathbb{R}$ implies  $a \in \mathbb{R}$,
but $g(a) \in \mathbb{R}$ and the function compositon of $g(g(a)) \in \mathbb{R}$ implies $g(a) + g(g(a))\neq 2a$ \newline
since the scaling a by $2$ where $2$ , $a \in\mathbb{N}$ does not imply equality despite $\mathbb{N} \subset \mathbb{R}$ \newline
this contradiction it makes the function$ g:\mathbb{R}\longrightarrow \mathbb{R}$ \newline
undefined in the ordered field $\mathbb{R}$ and this 
completes the proof. $\blacksquare$

\newpage

\section{Question 2}

\begin{center}
    \fontsize{24pt}{10pt}\selectfont
    \textsc{\textbf{(a)}}
\end{center}

let $A,B,C \in \mathbb{E}^n$ such that the $\triangle ABC$ is defined considering point A as the origin.
Lines $AB$ ,$AC$ and $BC$ are collinear thats is $AB ,AC ,BC \in  \mathbb{R}^n$. Point $Q$ is an interior point of $\triangle ABC$
iff there exists an open ball completely contained in set $A,B,C \in \mathbb{E}^n$ (inscribed circle in $\triangle ABC$)
For equality of $\angle QCB$ and $\angle QAC$ , $\angle QBC$ and  $\angle QAB$ there should exist a line of symmetry that run
from any vertix point on the $\triangle ABC$ for a metirc $d \in \mathbb{R}^n$, $d$ 
being the euclidian distance function on $\mathbb{R}^n$  in   $\mathbb{E}^n$ such that under the symmetry $d(AC) = d(AB) = d(BC)$
and the symmetry line is collinear with point $Q$ and the verticies $A ,B ,C$ of $\triangle ABC$ 
are othogonal to interior point $Q$ $\in \triangle ABC$ subset of ball centered at $Q$ with $A, B, C$ as its boundary points
With that you release that 
\begin{center}
    $\cos (\angle QCB) = \cos (\angle QAC)$ \newline
    $\frac{(Q-C)\cdot (B-C)}{\left\lvert Q-C \right\rvert \left\lvert B-C \right\rvert }$ = 
    $\frac{(Q-A)\cdot (C-A)}{\left\lvert Q-A \right\rvert \left\lvert C-A \right\rvert }$
    Because of symmetry $d(QC) = \left\lvert Q-C \right\rvert = \left\lvert Q-A \right\rvert$ and\newline
    $d(CA) = \left\lvert C-A  \right\rvert = \left\lvert B-C \right\rvert$ then expressing \newline
    $\cos\angle QAC$ in form of $\cos\angle QCB$ \newline
    $\cos\angle QCB$ $ = \frac{(Q-A)\cdot (C-A)}{\left\lvert Q-A \right\rvert \left\lvert C-A \right\rvert }$ \newline
\end{center}
given that $d(QA) = $$d(QC)$ and $d(BC) = $$d(CA)$ by symmetry.
The same applies to the $\angle QBC$ and $\angle QAB$ since symmetry is an isometric transformation that preserve distances.
and this completes the proof $\blacksquare$

\newpage

\section{Question 3}

\end{document}