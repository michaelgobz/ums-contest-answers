\documentclass[a4paper, 12pt]{article}
\usepackage[T1]{fontenc}
\usepackage[utf8]{inputenc}
\usepackage{geometry}
\usepackage{fancyhdr}
\usepackage{amssymb}

\pagestyle{fancy}
\fancyhf{}
\rhead{Makerere Univeristy | Kampala }
\lhead{\bfseries Michael Goboola  09/29/2000  Male }
\rfoot{}


\begin{document}

\begin{center}
    \fontsize{24pt}{10pt}\selectfont
    \textsc{\textbf{Attempted}}
\end{center}

\section{Question 1}
a  $\in$  $\mathbb{R}$, b  $\in$  $\mathbb{R}$ then by  the mapping g , g(a)  $\in$ $\mathbb{R}$ and
g(b) $\in$ $\mathbb{R}$. Suppose a,b $\in$ $\mathbb{N}$ and $\mathbb{N}$ $\subset$ $\mathbb{R}$, 
then by the principle of recursive definition there exists a function $\phi$ such that
$\phi$(a + b) = $\phi$(a) + $\phi$(b) and $\phi$(ab) = $\phi$(a)$\phi$(b).
If a = b , the g(a)^2 $\equiv$ g(a)g(a) $\equiv$ g(a^2)  and g(a$^$2 + a + g(a)) $\equiv$ g(a$^$2) + g(a) + g(g(a)).
Then proceeding with contradiction taking from the problem we have:
\begin{center}
    g(a$^$2 + b + g(b)) = 2b + g(a)$^$2
    taking b = a, $\forall$ a,b $\in$ $\mathbb{N}$ $\subset$ $\mathbb{R}$
    by the principle of recursive definition
    g(a)g(a) + g(a) + g(g(a)) = 2a + g(a)g(a)
    g(a) + g(g(a)) = 2a
\end{center}
but since a $\in$ $\mathbb{N}$ and $\mathbb{N}$ $\subset$ $\mathbb{R}$ $\implies $ a $\in$ $\mathbb{R}$, 
but g(a) $\in$ $\mathbb{R}$ and the function compositon of g(g(a)) $\in$ $\mathbb{R}$ $\implies $ g(a) + g(g(a)) $\neq$  2a
since the scaling a by 2 where 2 , a $\in$ $\mathbb{N}$ does not imply equality despite $\mathbb{N}$ $\subset$ $\mathbb{R}$ 
this contradiction it makes the function g:$\mathbb{R}$ $\longrightarrow$ $\mathbb{R}$ undefined in the ordered field $\mathbb{R}$ completes the proof. $\blacksquare$ 

\newpage

\section{Question 2}

\newpage

\section{Question 3}

\end{document}